\documentclass{article}

% Imports
\usepackage[a4paper, total={6in, 8in}]{geometry} % Imposta i margini della pagina
\usepackage{array} % Fornisce per le tabelle il tipo di colonna "m", che consente di impostare una larghezza fissa
\usepackage{tabularray} % Fornisce per le tabelle i tipi di colonna che permettono l'allineamento verticale
\usepackage{booktabs} % Fornisce decoratori per le tabelle
\usepackage{graphicx} % Inserimento immagini

% Use case descriptions imports & settings
\usepackage[dvipsnames, table]{xcolor} % Per colorare celle delle tabelle
\usepackage{tabularx}
\usepackage{enumitem}
\setlist[enumerate]{label*=\arabic*., topsep=0pt, itemsep=0pt, parsep=0pt, partopsep=0pt} % Imposta gli indici di tutti i livelli di enumerazione come numeri arabi (invece del default che passa alle lettere) e rimuove il padding
\usepackage{float} % Fornisce la posizione 'H' per le tabelle, che le posiziona direttamente dove specificato nel codice latex e non in cima alla pagina o altri comportamenti 'float' non desiderati

% Macros
\newcommand{\mycount}[1]{\stepcounter{#1}\arabic{#1}}

\title{Tell: Gestore teatrale}
\date{2025-06-20}
\author
{
    Giuseppe Mariotti \\
    Riccardo Ripa \\
    Rodrigo Rengifo \\
    Mattia Praolini
}

\begin{document}
    \pagenumbering{gobble}
    \maketitle
    \newpage
    \pagenumbering{arabic}

    \section{Panoramica}
        Il progetto prevede la realizzazione di un applicativo che consenta la vendita di biglietti per eventi teatrali (come opere e concerti) e la divulgazione di informazioni a riguardo. Il sistema si articolerà, per l’acquirente, in due sezioni: la prima conterrà la lista di eventi disponibili all'acquisto, la seconda conterrà maggiori informazioni riguardo le opere, incluse quelle non in programma al momento. Per la biglietteria saranno disponibili sezioni per la gestione degli eventi e delle prenotazioni, mentre per l'amministratore saranno disponibili sezioni per la gestione delle informazioni riguardo le opere e della struttura posti del teatro. Dall’intervista con i responsabili marketing e ufficio stampa di un ente pesarese operante nel settore sono emerse varie necessità. L’interfaccia dovrà essere ottimizzata per favorire la semplicità del processo di acquisto del biglietto: un’eccessiva quantità di informazioni in primo accesso ridurrebbe l’intuitività e l’accessibilità. Il prezzo di un biglietto è determinato dalla tipologia di posto scelto e dal tipo di evento, nonché da una lista di sconti applicabili in base a determinati criteri. Una volta confermata la prenotazione, il totem digitale stamperà una ricevuta contenente l'identificativo che potrà essere usato per effettuare il pagamento in biglietteria. Le voci delle informazioni dell’evento (direttore d’orchestra, interpreti, tecnici e altri) dovranno essere totalmente flessibili viste le notevoli differenze fra i vari tipi di eventi. Un caso particolare di eventi sono le opere, che possono essere messe in scena solo tramite una regia (una direzione artistica che accoppia alle musiche e al libretto dell’opera i costumi, fondali e altri elementi necessari alla rappresentazione). Infine, nella sezione anagrafica, per ogni opera dovrà essere memorizzato e mostrato all’utente lo storico delle regie presenti e passate.
    \newpage

    \section{Glossario dei termini}
        \begin{tblr}{h{3cm}h{2cm}m{10cm}}
            \hline
            \textbf{Termine} & \textbf{Sinonimi} & \textbf{Definizione} \\
            \hline
            Opera & N/A & Il testo e le musiche di un'opera lirica. Questi, a meno di casi li- mite che non interessano la realizzazione del software in questione, non vengono mai modificati. \\
            Regia & N/A & I costumi, i fondali, la direzione coreografica e degli effetti speciali e, in generale, della narrazione visiva dell'opera. Periodicamente le opere ricevono una nuova regia, che viene realizzata da un regista. \\
            Spettacolo & N/A & Le informazioni organizzative riguardo una determinata rappresentazione teatrale (di una \textbf{Regia} o altro tipo di spettacolo), come il cast, il direttore o il coro. \\
            Evento & N/A & La data e l'ora in cui un determinato spettacolo viene messo in scena. Solitamente, ciascuno spettacolo viene ripetuto in molteplici eventi. \\
            Cliente & N/A & L'aquirente di biglietti \\
            Biglietteria & N/A & Il dipendente dell'ufficio biglietteria \\
            Amministratore & N/A & Il gestore del sistema \\
            Utente & N/A & Uno qualunque tra \textbf{Cliente}, \textbf{Biglietteria} e \textbf{Amministratore} \\
            \hline
        \end{tblr}
    \newpage

    \section{Requisiti progettuali}
        \subsection{Diagramma}
            \includegraphics{imgs/requisiti/requisiti}

        \subsection{Requisiti Funzionali}
            \newcounter{rf}

            \subsubsection{Area Cliente}
                \textbf{RF\mycount{rf}: Visualizza opere} \\
                Il sistema dovrà mostrare al cliente le anagrafiche delle opere e l'archivio delle relative regie. \\
                \\
                \textbf{RF\mycount{rf}: Visualizza spettacoli} \\
                Il sistema dovrà mostrare al cliente la lista degli spettacoli e dei relativi eventi disponibili all'acquisto. \\
                \\
                \textbf{RF\mycount{rf}: Visualizza struttura posti} \\
                Il sistema dovrà consentire al cliente di visualizzare la struttura posti del teatro durante la composizione della prenotazione. \\
                \\
                \textbf{RF\mycount{rf}: Gestisci prenotazione} \\
                Il sistema dovrà consentire al cliente di creare una prenotazione e modificarne i posti o eliminarla completamente prima di averla confermata. \\
                \\
                \textbf{RF\mycount{rf}: Conferma prenotazione} \\
                Il sistema dovrà consentire al cliente di confermare la prenotazione, e produrre la ricevuta contenente l'identificativo della prenotazione per il pagamento alla cassa. \\

            \subsubsection{Area Biglietteria}
                \textbf{RF\mycount{rf}: Gestisci spettacoli} \\
                Il sistema dovrà consentire alla biglietteria di creare, modificare e rimuovere gli spettacoli in programma e i relativie eventi. \\
                \\
                \textbf{RF\mycount{rf}: Gestisci lista prenotazioni} \\
                Il sistema dovrà consentire alla biglietteria di visualizzare e modificare tutte le prenotazioni effettuate nel teatro e di contrassegnare le prenotazioni che sono state pagate in cassa. \\

            \subsubsection{Area Amministratore}
                \textbf{RF\mycount{rf}: Gestisci opere} \\
                Il sistema dovrà consentire all'amministratore di creare, modificare e rimuovere le anagrafiche delle opere e l'archivio delle relative regie. \\
                \\
                \textbf{RF\mycount{rf}: Gestisci struttura posti} \\
                Il sistema dovrà consentire all'amministratore di modificare la struttura posti del teatro. \\

        \subsection{Requisiti Non Funzionali}
            \newcounter{rnf}

            \textbf{RNF\mycount{rnf}: Python3} \\
            Il sistema dovrà essere implementato in linguaggio Python3. \\

    \section{Casi d'uso}
        \subsection{Attori}
            \subsubsection{Diagramma}
                \includegraphics{imgs/use_case/attori}

        \subsection{Cliente}
            \subsubsection{Diagramma}
                \includegraphics[width=\textwidth]{imgs/use_case/cliente}

            \subsubsection{Descrizioni}
                % UC: Lista Opere
                \begin{table}[H]
                    \begin{tabular}{|p{\linewidth}|}
                        \hline
                        \cellcolor{gray!100}
                        \color{white}
                        \centerline{\textbf{Caso d'uso:} Lista Opere} \\
                        \hline
                        \textbf{ID:} 1 \\
                        \hline
                        \cellcolor{gray!20}
                        \textbf{Breve descrizione:} \\
                        \cellcolor{gray!20}
                        L'utente ottiene la lista di opere memorizzate nel sistema. \\
                        \hline
                        \textbf{Attori primari:} \\
                        \begin{minipage}{\linewidth}
                            Cliente \\
                            Biglietteria \\
                            Amministratore
                        \end{minipage}
                        \vspace{0pt} \\  % Utilizzo \vspace{} per regolare lo spazio tra le sezioni delle tabelle.
                        \hline
                        \textbf{Attori secondari:} \\
                        Nessuno. \\
                        \hline
                        \cellcolor{gray!20}
                        \textbf{Precondizioni:} \\
                        \cellcolor{gray!20}
                        \begin{minipage}{\linewidth}
                            \begin{enumerate}
                                \item L'utente è posizionato nella sezione informazioni.
                            \end{enumerate}
                        \end{minipage} \\
                        \hline
                        \textbf{Sequenza degli eventi principale:}
                        \begin{enumerate}
                            \item Il caso d'uso inizia quando l'utente entra nella sezione informazioni.
                            \item \textbf{If} nel sistema è memorizzata almeno un'opera
                            \begin{enumerate}
                                \item \textbf{For} ogni opera nel sistema
                                \begin{enumerate}
                                    \item[] \textbf{punto di estensione:} filtro ricerca
                                    \item \textbf{include}(Visualizza Opera)
                                \end{enumerate}
                            \end{enumerate}
                            \item \textbf{Else}
                            \begin{enumerate}
                                \item Il sistema informa l'utente che non vi sono opere memorizzate.
                            \end{enumerate}
                        \end{enumerate} \\
                        \hline
                        \cellcolor{gray!20}
                        \textbf{Postcondizioni:} \\
                        \cellcolor{gray!20}
                        \begin{minipage}{\linewidth}
                            \begin{enumerate}
                                \item Viene visualizzata a schermo la lista delle opere memorizzate.
                            \end{enumerate}
                        \end{minipage} \\
                        \hline
                        \textbf{Sequenza degli eventi alternativa:} \\
                        Nessuna. \\
                        \hline
                    \end{tabular}
                \end{table}

                % UC EXT: Ricerca Opera
                \begin{table}[H]
                    \begin{tabular}{|p{\linewidth}|}
                        \hline
                        \cellcolor{gray!100}
                        \color{white}
                        \centerline{\textbf{Caso d'uso di estensione:} Ricerca Opera} \\
                        \hline
                        \textbf{ID:} 2 \\
                        \hline
                        \cellcolor{gray!20}
                        \textbf{Breve descrizione:} \\
                        \cellcolor{gray!20}
                        \textbf{Segmento 1:} L'utente filtra la lista di opere tramite parola chiave ricercata nel nome. \\
                        \hline
                        \textbf{Attori primari:} \\
                        \begin{minipage}{\linewidth}
                            Cliente \\
                            Biglietteria \\
                            Amministratore
                        \end{minipage}
                        \vspace{0pt} \\  % Utilizzo \vspace{} per regolare lo spazio tra le sezioni delle tabelle.
                        \hline
                        \textbf{Attori secondari:} \\
                        Nessuno. \\
                        \hline
                        \cellcolor{gray!20}
                        \textbf{Precondizioni del segmento 1:} \\
                        \cellcolor{gray!20}
                        \begin{minipage}{\linewidth}
                            \begin{enumerate}
                                \item L'utente ha inserito una chiave di ricerca.
                                \item L'utente ha premuto il pulsante di ricerca.
                            \end{enumerate}
                        \end{minipage} \\
                        \hline
                        \textbf{Sequenza degli eventi principale del segmento 1:}
                        \begin{enumerate}
                            \item Le opere che non contengono la parola chiave vengono eliminate dalla lista risultante.
                        \end{enumerate} \\
                        \hline
                        \cellcolor{gray!20}
                        \textbf{Postcondizioni del segmento 1:} \\
                        \cellcolor{gray!20}
                        \begin{minipage}{\linewidth}
                            \begin{enumerate}
                                \item La lista risultante è filtrata.
                            \end{enumerate}
                        \end{minipage} \\
                        \hline
                    \end{tabular}
                \end{table}

                % UC: Visualizza Opera
                \begin{table}[H]
                    \begin{tabular}{|p{\linewidth}|}
                        \hline
                        \cellcolor{gray!100}
                        \color{white}
                        \centerline{\textbf{Caso d'uso:} Visualizza Opera} \\
                        \hline
                        \textbf{ID:} 3 \\
                        \hline
                        \cellcolor{gray!20}
                        \textbf{Breve descrizione:} \\
                        \cellcolor{gray!20}
                        L'utente ottiene le informazioni anagrafiche riguardo una singola opera. \\
                        \hline
                        \textbf{Attori primari:} \\
                        \begin{minipage}{\linewidth}
                            Cliente \\
                            Biglietteria \\
                            Amministratore
                        \end{minipage}
                        \vspace{0pt} \\  % Utilizzo \vspace{} per regolare lo spazio tra le sezioni delle tabelle.
                        \hline
                        \textbf{Attori secondari:} \\
                        Nessuno. \\
                        \hline
                        \cellcolor{gray!20}
                        \textbf{Precondizioni:} \\
                        \cellcolor{gray!20}
                        \begin{minipage}{\linewidth}
                            \begin{enumerate}
                                \item L'utente è posizionato nella sezione informazioni.
                            \end{enumerate}
                        \end{minipage} \\
                        \hline
                        \textbf{Sequenza degli eventi principale:}
                        \begin{enumerate}
                            \item Il caso d'uso inizia quando l'utente richiede le informazioni di un'opera.
                            \item Il sistema cerca le informazioni anagrafiche nella memoria.
                            \item Il sistema mostra le informazioni a schermo.
                            \item \textbf{include}(Lista Regie)
                            \item[] \textbf{punto di estensione:} modifica
                        \end{enumerate} \\
                        \hline
                        \cellcolor{gray!20}
                        \textbf{Postcondizioni:} \\
                        \cellcolor{gray!20}
                        \begin{minipage}{\linewidth}
                            \begin{enumerate}
                                \item Sono visualizzate a schermo le informazioni anagrafiche dell'opera.
                            \end{enumerate}
                        \end{minipage} \\
                        \hline
                        \textbf{Sequenza degli eventi alternativa:} \\
                        Nessuna. \\
                        \hline
                    \end{tabular}
                \end{table}

                % UC: Lista Regie
                \begin{table}[H]
                    \begin{tabular}{|p{\linewidth}|}
                        \hline
                        \cellcolor{gray!100}
                        \color{white}
                        \centerline{\textbf{Caso d'uso:} Lista Regie} \\
                        \hline
                        \textbf{ID:} 4 \\
                        \hline
                        \cellcolor{gray!20}
                        \textbf{Breve descrizione:} \\
                        \cellcolor{gray!20}
                        L'utente ottiene la lista di regie relative all'opera che sta visualizzando. \\
                        \hline
                        \textbf{Attori primari:} \\
                        \begin{minipage}{\linewidth}
                            Cliente \\
                            Biglietteria \\
                            Amministratore
                        \end{minipage}
                        \vspace{0pt} \\  % Utilizzo \vspace{} per regolare lo spazio tra le sezioni delle tabelle.
                        \hline
                        \textbf{Attori secondari:} \\
                        Nessuno. \\
                        \hline
                        \cellcolor{gray!20}
                        \textbf{Precondizioni:} \\
                        \cellcolor{gray!20}
                        \begin{minipage}{\linewidth}
                            \begin{enumerate}
                                \item L'utente sta visualizzando una singola opera.
                            \end{enumerate}
                        \end{minipage} \\
                        \hline
                        \textbf{Sequenza degli eventi principale:}
                        \begin{enumerate}
                            \item Il caso d'uso inizia quando l'utente visualizza una singola opera.
                            \item \textbf{If} per l'opera è memorizzata almeno una regia
                            \begin{enumerate}
                                \item \textbf{For} ogni regia dell'opera
                                \begin{enumerate}
                                    \item Il sistema mostra le informazioni della regia a schermo.
                                \end{enumerate}
                            \end{enumerate}
                            \item \textbf{Else}
                            \begin{enumerate}
                                \item Il sistema informa l'utente che non vi sono regie memorizzate.
                            \end{enumerate}
                        \end{enumerate} \\
                        \hline
                        \cellcolor{gray!20}
                        \textbf{Postcondizioni:} \\
                        \cellcolor{gray!20}
                        \begin{minipage}{\linewidth}
                            \begin{enumerate}
                                \item E' visualizzata a schermo la lista delle regie associate all'opera.
                            \end{enumerate}
                        \end{minipage} \\
                        \hline
                        \textbf{Sequenza degli eventi alternativa:} \\
                        Nessuna. \\
                        \hline
                    \end{tabular}
                \end{table}

                % UC: Crea Prenotazione
                \begin{table}[H]
                    \begin{tabular}{|p{\linewidth}|}
                        \hline
                        \cellcolor{gray!100}
                        \color{white}
                        \centerline{\textbf{Caso d'uso:} Crea Prenotazione} \\
                        \hline
                        \textbf{ID:} 12 \\
                        \hline
                        \cellcolor{gray!20}
                        \textbf{Breve descrizione:} \\
                        \cellcolor{gray!20}
                        L'utente prenota posti di un evento svolto nel teatro. \\
                        \hline
                        \textbf{Attori primari:} \\
                        \begin{minipage}{\linewidth}
                            Cliente \\
                            Biglietteria \\
                            Amministratore
                        \end{minipage}
                        \vspace{0pt} \\
                        \hline
                        \textbf{Attori secondari:} \\                        
                        Nessuno. \\
                        \hline
                        \cellcolor{gray!20}
                        \textbf{Precondizioni:} \\
                        \cellcolor{gray!20}
                        \begin{minipage}{\linewidth}
                            \begin{enumerate}
                                \item L'utente visualizza l'elenco degli eventi. %// Non ne sono convinto.
                            \end{enumerate}
                        \end{minipage} \\
                        \hline
                        \textbf{Sequenza degli eventi principale:}
                        \begin{enumerate}
                            \item Il caso d'uso inizia quando l'utente preme il pulsante \emph{Crea prenotazione}.
                            \item Il sistema mostra una interfaccia per inserire i dati relativi alla prenotazione.
                            \item L'utente inserisce i dati relativi alla prenotazione.
                            \item L'utente preme il pulsante \emph{Salva dati}.
                            \item \textbf{If} i dati inseriti sono validi
                            \begin{enumerate}
                                \item Il sistema salva i dati.
                                \item \textbf{If} l'utente è Cliente
                                \begin{enumerate}
                                    \item L'utente è inviato alla categoria "Prenotazioni".
                                \end{enumerate}
                                \item Il caso d'uso finisce.
                            \end{enumerate}
                            \item \textbf{Else}
                            \begin{enumerate}
                                \item Il sistema informa che i dati inseriti sono invalidi.
                                \item L'utente torna al passo 3.
                            \end{enumerate}
                        \end{enumerate} \\
                        \hline
                        \cellcolor{gray!20}
                        \textbf{Postcondizioni:} \\
                        \cellcolor{gray!20}
                        \begin{minipage}{\linewidth}
                            \begin{enumerate}
                                \item L'utente crea una prenotazione.
                                \item Se l'utente è Cliente, è inviato alla categoria "Prenotazioni".
                            \end{enumerate}
                        \end{minipage}
                        \vspace{-5pt} \\
                        \hline
                        \textbf{Sequenza degli eventi alternativa:} \\
                        Nessuna. \\
                        \hline
                    \end{tabular}
                \end{table}

                % UC: Visualizza Prenotazione
                \begin{table}[H]
                    \begin{tabular}{|p{\linewidth}|}
                        \hline
                        \cellcolor{gray!100}
                        \color{white}
                        \centerline{\textbf{Caso d'uso:} Visualizza Prenotazione} \\
                        \hline
                        \textbf{ID:} 13 \\
                        \hline
                        \cellcolor{gray!20}
                        \textbf{Breve descrizione:} \\
                        \cellcolor{gray!20}
                        L'utente ottiene l'informazione anagrafica riguardo una singola prenotazione. \\
                        \hline
                        \textbf{Attori primari:} \\
                        \begin{minipage}{\linewidth}
                            Cliente \\
                            Biglietteria \\
                            Amministratore
                        \end{minipage}
                        \vspace{0pt} \\
                        \hline
                        \textbf{Attori secondari:} \\
                        Nessuno. \\
                        \hline
                        \cellcolor{gray!20}
                        \textbf{Precondizioni:} \\
                        \cellcolor{gray!20}
                        \begin{minipage}{\linewidth}
                            \begin{enumerate}
                                \item L'utente si trova nella categoria "Prenotazioni". % Utilizo "categoria" per non confonderla con "sezione".
                                \item L'elenco delle prenotazioni, sia esso filtrato o meno, non è vuoto.
                            \end{enumerate}
                        \end{minipage}
                        \vspace{-5pt} \\
                        \hline
                        \textbf{Sequenza degli eventi principale:}
                        \begin{enumerate}
                            \item Il caso d'uso inizia quando l'utente richiede le informazioni di una prenotazione.
                            \item Il sistema mostra le informazioni relative alla prenotazione.
                            \item Il sistema mostra tre pulsanti vincolati alla prenotazione: \emph{Modifica}, \emph{Elimina} e \emph{Genera ricevuta}.
                            \item \textbf{If} l'utente è Amministratore o Biglietteria
                            \begin{enumerate}
                                \item Il sistema mostra il pulsante \emph{Segnare}.
                            \end{enumerate}
                            \item[] \textbf{punto di estensione:} modificaPrenotazione
                            %// Per praticità, Genera Ricevuta dovrebbe essere un caso incluso Visualizza Prenotazione.
                        \end{enumerate} \\
                        \hline
                        \cellcolor{gray!20}
                        \textbf{Postcondizioni:} \\
                        \cellcolor{gray!20}
                        \begin{minipage}{\linewidth}
                            \begin{enumerate}
                                \item L'utente visualizza le informazioni di una prenotazione.
                                \item Il sistema permette all'utente di modificare, eliminare o generare una ricevuta della prenotazione.  % Molto dettagliato :/
                                \item Se l'utente è Amministratore o Biglietteria, il sistema permette di segnarla come \emph{pagata}.
                            \end{enumerate}
                        \end{minipage}
                        \vspace{0pt} \\
                        \hline
                        \textbf{Sequenza degli eventi alternativa:} \\
                        Nessuna. \\
                        \hline
                    \end{tabular}
                \end{table}

                % UC EXT: Modifica Prenotazione
                \begin{table}[H]
                    \begin{tabular}{|p{\linewidth}|}
                        \hline
                        \cellcolor{gray!100}
                        \color{white}
                        \centerline{\textbf{Caso d'uso dell'estensione:} Modifica Prenotazione} \\
                        \hline
                        \textbf{ID:} 14 \\
                        \hline
                        \cellcolor{gray!20}
                        \textbf{Breve descrizione:} \\
                        \cellcolor{gray!20}
                        Segmento 1: L'utente modifica una prenotazione esistente all'interno dell'elenco delle prenotazioni. \\
                        \hline
                        \textbf{Attori principali:} \\
                        \begin{minipage}{\linewidth}
                            Cliente \\
                            Biglietteria \\
                            Amministatore
                        \end{minipage}
                        \vspace{0pt} \\
                        \hline
                        \textbf{Attori secondari:} \\                        
                        Nessuno. \\
                        \hline
                        \cellcolor{gray!20}
                        \textbf{Precondizioni del segmento 1:} \\
                        \cellcolor{gray!20}
                        \begin{minipage}{\linewidth}
                            \begin{enumerate}
                                \item L'utente visualizza l'elenco delle prenotazioni.
                                \item L'utente ha premuto il pulsante di modifica.
                            \end{enumerate}
                        \end{minipage}
                        \vspace{-5pt} \\
                        \hline
                        \textbf{Sequenza degli eventi del segmento 1:}
                        \begin{enumerate}
                            \item Il sistema mostra la interfaccia di creazione di prenotazione, ma con i dati saltavi della prenotazione già creata selezionati. %// Da confirmare coi ragazzi.
                            \item L'utente modifica i dati relativi alla prenotazione.
                            \item L'utente preme il pulsante \emph{Salva dati}.
                            \item \textbf{If} i dati sono validi
                            \begin{enumerate}
                                \item Il sistema salva i dati.
                                \item Il caso d'uso finisce.
                            \end{enumerate}
                            \item \textbf{Else}
                            \begin{enumerate}
                                \item Il sistema informa che i dati inseriti sono invalidi.
                                \item L'utente torna al passo 2.
                            \end{enumerate}
                        \end{enumerate} \\
                        \hline
                        \cellcolor{gray!20}
                        \textbf{Postcondizioni del segmento 1:} \\
                        \cellcolor{gray!20}
                        \begin{minipage}{\linewidth}
                            \begin{enumerate}
                                \item L'utente effettua la modifica di una prenotazione.
                            \end{enumerate}
                        \end{minipage} \\
                        \hline
                    \end{tabular}
                \end{table}

                % UC EXT: Elimina Prenotazione
                \begin{table}[H]
                    \begin{tabular}{|p{\linewidth}|}
                        \hline
                        \cellcolor{gray!100}
                        \color{white}
                        \centerline{\textbf{Caso d'uso dell'estensione:} Elimina Prenotazione} \\
                        \hline
                        \textbf{ID:} 15 \\
                        \hline
                        \cellcolor{gray!20}
                        \textbf{Breve descrizione:} \\
                        \cellcolor{gray!20}
                        Segmento 1: L'utente elimina una prenotazione esistente all'interno dell'elenco delle prenotazioni. \\
                        \hline
                        \textbf{Attori principali:} \\
                        \begin{minipage}{\linewidth}
                            Cliente \\
                            Biglietteria \\
                            Amministratore
                        \end{minipage}
                        \vspace{0pt} \\
                        \hline
                        \textbf{Attori secondari:} \\                        
                        Nessuno. \\
                        \hline
                        \cellcolor{gray!20}
                        \textbf{Precondizioni del segmento 1:} \\
                        \cellcolor{gray!20}
                        \begin{minipage}{\linewidth}
                            \begin{enumerate}
                                \item L'utente visualizza l'elenco delle prenotazioni.
                                \item L'utente ha premuto il pulsante di eliminazione.
                            \end{enumerate}
                        \end{minipage}
                        \vspace{-5pt} \\
                        \hline
                        \textbf{Sequenza degli eventi del segmento 1:}
                        \begin{enumerate}
                            \item Il sistema chiede conferma per eliminare la prenotazione.
                            \item \textbf{If} l'utente preme il pulsante di conferma
                            \begin{enumerate}
                                \item Il sistema effettua l'eliminazione della prenotazione.
                                \item Il sistema torna all'elenco delle prenotazioni.
                                \item Il sistema informa che l'eliminazione è stata effettuata.
                            \end{enumerate}
                        \end{enumerate} \\
                        \hline
                        \cellcolor{gray!20}
                        \textbf{Postcondizioni del segmento 1:} \\
                        \cellcolor{gray!20}
                        \begin{minipage}{\linewidth}
                            \begin{enumerate}
                                \item L'utente effettua l'eliminazione di una prenotazione.
                            \end{enumerate}
                        \end{minipage} \\
                        \hline
                    \end{tabular}
                \end{table}

                % UC: Genera Ricevuta
                \begin{table}[H]
                    \begin{tabular}{|p{\linewidth}|}
                        \hline
                        \cellcolor{gray!100}
                        \color{white}
                        \centerline{\textbf{Caso d'uso:} Genera Ricevuta} \\
                        \hline
                        \textbf{ID:} 16 \\
                        \hline
                        \cellcolor{gray!20}
                        \textbf{Breve descrizione:} \\
                        \cellcolor{gray!20}
                        L'utente genera la ricevuta di una prenotazione e la stampa. \\
                        \hline
                        \textbf{Attori primari:} \\
                        \begin{minipage}{\linewidth}
                            Cliente \\
                            Biglietteria \\
                            Amministratore
                        \end{minipage}
                        \vspace{0pt} \\
                        \hline
                        \textbf{Attori secondari:} \\
                        Nessuno. \\
                        \hline
                        \cellcolor{gray!20}
                        \textbf{Precondizioni:} \\
                        \cellcolor{gray!20}
                        \begin{minipage}{\linewidth}
                            \begin{enumerate}
                                \item L'utente visualizza l'elenco delle prenotazioni. %// Da confrontare col diagramma dei casi d'uso dei Clienti.
                            \end{enumerate}
                        \end{minipage} \\
                        \hline
                        \textbf{Sequenza degli eventi principale:}
                        \begin{enumerate}
                            \item Il caso d'uso inizia quando l'utente preme \emph{Genera ricevuta}.
                            \item Il sistema genera un documento con tutta l'informazione relativa all'acquisto dei biglietti prenotati.
                            \item Il sistema stampa il documento.
                        \end{enumerate} \\
                        \hline
                        \cellcolor{gray!20}
                        \textbf{Postcondizioni:} \\
                        \cellcolor{gray!20}
                        \begin{minipage}{\linewidth}
                            \begin{enumerate}
                                \item La ricevuta è stata stampata.
                            \end{enumerate}
                        \end{minipage} \\
                        \hline
                        \textbf{Sequenza degli eventi alternativa:} \\
                        Nessuna. \\
                        \hline
                    \end{tabular}
                \end{table}

        \subsection{Biglietteria}
            \subsubsection{Diagramma}
                \includegraphics[width=\textwidth]{imgs/use_case/biglietteria}
            
            \subsubsection{Descrizioni}
                % UC: Lista Prenotazione
                \begin{table}[H]
                    \begin{tabular}{|p{\linewidth}|}
                        \hline
                        \cellcolor{gray!100}
                        \color{white}
                        \centerline{\textbf{Caso d'uso:} Lista Prenotazione} \\
                        \hline
                        \textbf{ID:} 23 \\
                        \hline
                        \cellcolor{gray!20}
                        \textbf{Breve descrizione:} \\
                        \cellcolor{gray!20}                        
                        L'utente visualizza l'elenco delle prenotazione salvate nel sistema. \\
                        \hline
                        \textbf{Attori primari:} \\
                        \begin{minipage}{\linewidth}
                            Biglietteria \\
                            Amministratore
                        \end{minipage}
                        \vspace {-5pt} \\
                        \hline
                        \textbf{Attori secondari:} \\                        
                        Nessuno. \\
                        \hline
                        \cellcolor{gray!20}
                        \textbf{Precondizioni:} \\
                        \cellcolor{gray!20}
                        \begin{minipage}{\linewidth}
                            \begin{enumerate}
                                \item L'utente è stato autenticato come Amministratore o Biglietteria.
                                \item L'utente si trova nella categoria "Prenotazioni".
                            \end{enumerate}
                        \end{minipage}
                        \vspace{0pt} \\
                        \hline
                        \textbf{Sequenza degli eventi principale:}
                        \begin{enumerate}
                            \item Questo caso d'uso inizia dopo che l'utente seleziona la categoria "Prenotazioni".
                            \item \textbf{If} nel sistema è memorizzata almeno una prenotazione
                            \begin{enumerate}
                                \item \textbf{For} ogni prenotazione nel sistema
                                \begin{enumerate}
                                    \item \textbf{include}(Visualizza Prenotazione)
                                \end{enumerate}
                                \item \item[] \textbf{punto di estensione:} filtrareElenco
                                \item \textbf{For} ogni prenotazione nell'elenco visualizzato
                                \begin{enumerate}
                                    \item[] \textbf{punto di estensione:} modificaPrenotazione %// Da confirmare coi ragazzi
                                \end{enumerate}
                            \end{enumerate}
                            \item \textbf{Else}
                            \begin{enumerate}
                                \item Il sistema informa l'utente che non vi sono prenotazioni memorizzate.
                            \end{enumerate}
                        \end{enumerate} \\
                        \hline
                        \cellcolor{gray!20}
                        \textbf{Postcondizioni:} \\
                        \cellcolor{gray!20}
                        \begin{minipage}{\linewidth}
                            \begin{enumerate}
                                \item L'utente visualizza l'elenco delle prenotazioni salvate nel sistema.
                            \end{enumerate}  
                        \end{minipage}
                        \vspace{-10pt} \\
                        \hline
                        \textbf{Sequenza degli eventi alternativa:} \\
                        Nessuna. \\
                        \hline
                    \end{tabular}
                \end{table}

                % UC EXT: Ricerca Prenotazione
                \begin{table}[H]
                    \begin{tabular}{|p{\linewidth}|}
                        \hline
                        \cellcolor{gray!100}
                        \color{white}
                        \centerline{\textbf{Caso d'uso d'estensione:} Ricerca Prenotazione} \\
                        \hline
                        \textbf{ID:} 24 \\
                        \hline
                        \cellcolor{gray!20}
                        \textbf{Breve descrizione:} \\
                        \cellcolor{gray!20}
                        Segmento 1: Questo caso d'uso consente di filtrare l'elenco delle prenotazioni attraverso una ricerca. \\
                        \hline
                        \textbf{Attori primari:} \\
                        \begin{minipage}{\linewidth}
                            Biglietteria \\
                            Amministratore
                        \end{minipage}
                        \vspace{0pt} \\
                        \hline
                        \textbf{Attori secondari:} \\
                        Nessuno. \\
                        \hline
                        \cellcolor{gray!20}
                        \textbf{Precondizioni del segmento 1:} \\
                        \cellcolor{gray!20}
                        \begin{minipage}{\linewidth}
                            \begin{enumerate}
                                \item L'utente ha inserito i criteri di ricerca.
                                \item L'utente ha premuto il pulsante di ricerca.
                            \end{enumerate}
                        \end{minipage} \\
                        \hline
                        \textbf{Sequenza degli eventi del segmento 1:}
                        \begin{enumerate}
                            \item Il sistema ricerca, nell'elenco delle prenotazioni, delle prenotazioni coincidenti.
                            \item \textbf{If} il sistema trova almeno una prenotazione coincidente
                            \begin{enumerate}
                                \item \textbf{For} ogni prenotazione coincidente
                                \begin{enumerate}
                                    \item \textbf{include}(Visualizza Prenotazione).
                                \end{enumerate}
                            \end{enumerate}
                            \item \textbf{Else}
                            \begin{enumerate}
                                \item Il sistema informa che non vi sono prenotazioni coincidenti.
                            \end{enumerate}
                        \end{enumerate} \\
                        \hline
                        \cellcolor{gray!20}
                        \textbf{Postcondizioni del segmento 1:} \\
                        \cellcolor{gray!20}
                        \begin{minipage}{\linewidth}
                            \begin{enumerate}
                                \item L'utente visualizza l'elenco filtrato.
                            \end{enumerate}
                        \end{minipage} \\
                        \hline
                    \end{tabular}
                \end{table}

                % UC EXT: Segna Prenotazione come Pagata
                \begin{table}[H]
                    \begin{tabular}{|p{\linewidth}|}
                        \hline
                        \cellcolor{gray!100}
                        \color{white}
                        \centerline{\textbf{Caso d'uso dell'estensione:} Segna Prenotazione come Pagata} \\
                        \hline
                        \textbf{ID:} 25 \\
                        \hline
                        \cellcolor{gray!20}
                        \textbf{Breve descrizione:} \\
                        \cellcolor{gray!20}
                        Segmento 1: L'utente segna una prenotazione come \emph{pagata}. \\
                        \hline
                        \textbf{Attori principali:} \\
                        \begin{minipage}{\linewidth}
                            Biglietteria \\
                            Amministratore
                        \end{minipage}
                        \vspace{-5pt} \\
                        \hline
                        \textbf{Attori secondari:} \\
                        Nessuno. \\
                        \hline
                        \cellcolor{gray!20}
                        \textbf{Precondizioni del segmento 1:} \\
                        \cellcolor{gray!20}
                        \begin{minipage}{\linewidth}
                            \begin{enumerate}
                                \item L'utente visualizza l'elenco delle prenotazioni.
                                \item L'utente ha premuto il pulsante di assegnamento.
                            \end{enumerate}
                        \end{minipage}
                        \vspace{-5pt} \\
                        \hline
                        \textbf{Sequenza degli eventi del segmento 1:}
                        \begin{enumerate}
                            \item Il sistema chiede conferma per segnare la prenotazione come \emph{pagata}.
                            \item \textbf{If} L'utente preme il pulsante di conferma
                            \begin{enumerate}
                                \item Il sistema segna la prenotazione come \emph{pagata}.
                                \item Il sistema torna all'elenco delle prenotazioni.
                                \item Il sistema informa che l'assegnamento è stato effettuato.
                            \end{enumerate}
                        \end{enumerate} \\
                        \hline
                        \cellcolor{gray!20}
                        \textbf{Postcondizioni del segmento 1:} \\
                        \cellcolor{gray!20}
                        \begin{minipage}{\linewidth}
                            \begin{enumerate}
                                \item L'utente effettua l'assegnamento di una prenotazione come \emph{pagata}.
                            \end{enumerate}
                        \end{minipage} \\
                        \hline
                    \end{tabular}
                \end{table}

        \subsection{Amministratore}
            \subsubsection{Diagramma}
                \includegraphics[height=\textheight]{imgs/use_case/amministratore}

            \subsubsection{Descrizioni}
                % UC: Crea Opera
                \begin{table}[H]
                    \centering
                    \begin{tabular}{|p{\linewidth}|}
                        \hline
                        \cellcolor{gray!100}
                        \color{white}
                        \centerline{\textbf{Caso d'uso:} Crea Opera} \\
                        \hline
                        \textbf{ID:} 26 \\
                        \hline
                        \cellcolor{gray!20}
                        \textbf{Breve descrizione:} \\
                        \cellcolor{gray!20}
                        L'utente crea una istanza di un'opera svolta nel teatro. \\
                        \hline
                        \textbf{Attori primari:} \\
                        \begin{minipage}{\linewidth}
                            Amministratore
                        \end{minipage}
                        \vspace{-10pt} \\
                        \hline
                        \textbf{Attori secondari:} \\
                        Nessuno. \\
                        \hline
                        \cellcolor{gray!20}
                        \textbf{Precondizioni:} \\
                        \cellcolor{gray!20}
                        \begin{minipage}{\linewidth}
                            \begin{enumerate}
                                \item L'utente è stato autenticato come Amministratore
                                \item L'utente si trova nella categoria "Opere".
                            \end{enumerate}
                        \end{minipage}
                        \vspace{-5pt} \\
                        \hline
                        \textbf{Sequenza degli eventi principale:}
                        \begin{enumerate}
                            \item Questo caso d'uso inizia quando l'utente preme il pulsante \emph{Crea opera}.
                            \item Il sistema mostra una interfaccia per inserire dati e impostazioni relativi all'opera. % e.g. Disponibilità
                            \item L'utente inserisce le informazioni anagrafiche relative all'opera.
                            \item \textbf{If} i dati inseriti sono validi
                            \begin{enumerate}
                                \item Il sistema salva i dati e le impostazioni.
                                \item Il caso d'uso finisce.
                            \end{enumerate}
                            \item \textbf{Else}
                            \begin{enumerate}
                                \item Il sistema informa l'utente che i dati inseriti sono invalidi.
                                \item L'utente torna al passo 3.
                            \end{enumerate}
                        \end{enumerate} \\
                        \hline
                        \cellcolor{gray!20}
                        \textbf{Postcondizioni:} \\
                        \cellcolor{gray!20}
                        \begin{minipage}{\linewidth}
                            \begin{enumerate}
                                \item L'utente crea un'opera.
                            \end{enumerate}
                        \end{minipage}
                        \vspace{-10pt} \\
                        \hline
                        \textbf{Sequenza degli eventi alternativa:} \\
                        Nessuna. \\
                        \hline
                    \end{tabular}
                \end{table}

                % UC EXT: Modifica Opera
                \begin{table}[H]
                    \begin{tabular}{|p{\linewidth}|}
                        \hline
                        \cellcolor{gray!100}
                        \color{white}
                        \centerline{\textbf{Caso d'uso di estensione:} Modifica Opera} \\
                        \hline
                        \textbf{ID:} 27 \\
                        \hline
                        \cellcolor{gray!20}
                        \textbf{Breve descrizione:} \\
                        \cellcolor{gray!20}
                        \textbf{Segmento 1:} L'utente modifica le informazioni anagrafiche di un'opera. \\
                        \hline
                        \textbf{Attori primari:} \\
                        \begin{minipage}{\linewidth}
                            Amministratore
                        \end{minipage}
                        \vspace{-10pt} \\  % Utilizzo \vspace{} per regolare lo spazio tra le sezioni delle tabelle.
                        \hline
                        \textbf{Attori secondari:} \\
                        Nessuno. \\
                        \hline
                        \cellcolor{gray!20}
                        \textbf{Precondizioni del segmento 1:} \\
                        \cellcolor{gray!20}
                        \begin{minipage}{\linewidth}
                            \begin{enumerate}
                                \item L'utente e' autenticato con account amministratore.
                                \item L'utente ha selezionato un'opera.
                                \item L'utente ha premuto il pulsante di modifica.
                            \end{enumerate}
                        \end{minipage} \\
                        \hline
                        \textbf{Sequenza degli eventi principale del segmento 1:}
                        \begin{enumerate}
                            \item Il sistema consente all'utente di modificare le informazioni anagrafiche dell'opera. \label{start_mod_opera}
                            \item \textbf{include}(Crea Regia)
                            \item \textbf{include}(Modifica Regia)
                            \item \textbf{include}(Elimina Regia)
                            \item Il sistema richiede all'utente di confermare l'intenzione di apportare le modifiche.
                            \item \textbf{If} l'utente conferma
                                \begin{enumerate}
                                    \item Il sistema controlla la validità dei valori inseriti.
                                    \item \textbf{If} i valori sono validi 
                                        \begin{enumerate}
                                            \item Il sistema memorizza permanentemente le modifiche apportate.
                                        \end{enumerate}
                                    \item \textbf{Else}
                                        \begin{enumerate}
                                            \item Il sistema informa l'utente dell'errore.
                                            \item Il caso d'uso ricomincia da \ref{start_mod_opera}
                                        \end{enumerate}
                                \end{enumerate}
                            \item \textbf{Else}
                                \begin{enumerate}
                                    \item Il sistema ignora i cambiamenti apportati.
                                \end{enumerate}
                        \end{enumerate} \\
                        \hline
                        \cellcolor{gray!20}
                        \textbf{Postcondizioni del segmento 1:} \\
                        \cellcolor{gray!20}
                        \begin{minipage}{\linewidth}
                            \begin{enumerate}
                                \item Le modifiche apportate sono memorizzate e immediatamente disponibili a tutti gli utenti.
                            \end{enumerate}
                        \end{minipage} \\
                        \hline
                    \end{tabular}
                \end{table}

                % UC EXT: Elimina Opera
                \begin{table}[H]
                    \begin{tabular}{|p{\linewidth}|}
                        \hline
                        \cellcolor{gray!100}
                        \color{white}
                        \centerline{\textbf{Caso d'uso di estensione:} Elimina Opera} \\
                        \hline
                        \textbf{ID:} 28 \\
                        \hline
                        \cellcolor{gray!20}
                        \textbf{Breve descrizione:} \\
                        \cellcolor{gray!20}
                        \textbf{Segmento 1:} L'utente elimina un'opera dalla lista di opere memorizzate. \\
                        \hline
                        \textbf{Attori primari:} \\
                        \begin{minipage}{\linewidth}
                            Amministratore
                        \end{minipage}
                        \vspace{-10pt} \\  % Utilizzo \vspace{} per regolare lo spazio tra le sezioni delle tabelle.
                        \hline
                        \textbf{Attori secondari:} \\
                        Nessuno. \\
                        \hline
                        \cellcolor{gray!20}
                        \textbf{Precondizioni del segmento 1:} \\
                        \cellcolor{gray!20}
                        \begin{minipage}{\linewidth}
                            \begin{enumerate}
                                \item L'utente e' autenticato con account amministratore.
                                \item L'utente ha selezionato un'opera.
                                \item L'utente ha premuto il pulsante di eliminazione.
                            \end{enumerate}
                        \end{minipage} \\
                        \hline
                        \textbf{Sequenza degli eventi principale del segmento 1:}
                        \begin{enumerate}
                            \item Il sistema richiede all'utente di confermare l'intenzione di eliminare l'opera.
                            \item \textbf{If} l'utente conferma
                                \begin{enumerate}
                                    \item Il sistema elimina permanentemente l'opera dalla lista di opere memorizzate.
                                \end{enumerate}
                            \item \textbf{Else}
                                \begin{enumerate}
                                    \item Il sistema ignora la richiesta di eliminazione.
                                \end{enumerate}
                        \end{enumerate} \\
                        \hline
                        \cellcolor{gray!20}
                        \textbf{Postcondizioni del segmento 1:} \\
                        \cellcolor{gray!20}
                        \begin{minipage}{\linewidth}
                            \begin{enumerate}
                                \item Le modifiche apportate sono memorizzate e immediatamente disponibili a tutti gli utenti.
                            \end{enumerate}
                        \end{minipage} \\
                        \hline
                    \end{tabular}
                \end{table}

                % UC: Crea Regia
                \begin{table}[H]
                    \begin{tabular}{|p{\linewidth}|}
                        \hline
                        \cellcolor{gray!100}
                        \color{white}
                        \centerline{\textbf{Caso d'uso:} Crea Regia} \\
                        \hline
                        \textbf{ID:} 29 \\
                        \hline
                        \cellcolor{gray!20}
                        \textbf{Breve descrizione:} \\
                        \cellcolor{gray!20}
                        L'utente crea una nuova regia per un'opera. \\
                        \hline
                        \textbf{Attori primari:} \\
                        \begin{minipage}{\linewidth}
                            Amministratore
                        \end{minipage}
                        \vspace{-10pt} \\  % Utilizzo \vspace{} per regolare lo spazio tra le sezioni delle tabelle.
                        \hline
                        \textbf{Attori secondari:} \\
                        Nessuno. \\
                        \hline
                        \cellcolor{gray!20}
                        \textbf{Precondizioni:} \\
                        \cellcolor{gray!20}
                        \begin{minipage}{\linewidth}
                            \begin{enumerate}
                                \item L'utente ha premuto il pulsante di aggiunta di una nuova regia.
                            \end{enumerate}
                        \end{minipage} \\
                        \hline
                        \textbf{Sequenza degli eventi principale:}
                        \begin{enumerate}
                            \item Il caso d'uso inizia quando l'utente preme il pulsante di aggiunta di una nuova regia.
                            \item Il sistema richiede all'utente le informazioni della regia. \label{beginning}
                            \item \textbf{If} l'utente conferma
                                \begin{enumerate}
                                    \item Il sistema controlla la validità dei valori inseriti.
                                    \item \textbf{If} i valori sono validi 
                                        \begin{enumerate}
                                            \item Il sistema memorizza permanentemente la nuova regia.
                                        \end{enumerate}
                                    \item \textbf{Else}
                                        \begin{enumerate}
                                            \item Il sistema informa l'utente dell'errore.
                                            \item Il caso d'uso ricomincia da \ref{beginning}
                                        \end{enumerate}
                                \end{enumerate}
                            \item \textbf{Else}
                                \begin{enumerate}
                                    \item Il sistema non aggiunge la nuova regia.
                                \end{enumerate}
                        \end{enumerate} \\
                        \hline
                        \cellcolor{gray!20}
                        \textbf{Postcondizioni:} \\
                        \cellcolor{gray!20}
                        \begin{minipage}{\linewidth}
                            \begin{enumerate}
                                \item La nuova regia è memorizzata e immediatamente visibile a tutti gli utenti.
                            \end{enumerate}
                        \end{minipage} \\
                        \hline
                        \textbf{Sequenza degli eventi alternativa:} \\
                        Nessuna. \\
                        \hline
                    \end{tabular}
                \end{table}

                % UC: Modifica Regia
                \begin{table}[H]
                    \begin{tabular}{|p{\linewidth}|}
                        \hline
                        \cellcolor{gray!100}
                        \color{white}
                        \centerline{\textbf{Caso d'uso:} Modifica Regia} \\
                        \hline
                        \textbf{ID:} 30 \\
                        \hline
                        \cellcolor{gray!20}
                        \textbf{Breve descrizione:} \\
                        \cellcolor{gray!20}
                        L'utente modifica le informazioni di una regia. \\
                        \hline
                        \textbf{Attori primari:} \\
                        \begin{minipage}{\linewidth}
                            Amministratore
                        \end{minipage}
                        \vspace{-10pt} \\  % Utilizzo \vspace{} per regolare lo spazio tra le sezioni delle tabelle.
                        \hline
                        \textbf{Attori secondari:} \\
                        Nessuno. \\
                        \hline
                        \cellcolor{gray!20}
                        \textbf{Precondizioni:} \\
                        \cellcolor{gray!20}
                        \begin{minipage}{\linewidth}
                            \begin{enumerate}
                                \item L'utente ha premuto il pulsante di modifica di una regia.
                            \end{enumerate}
                        \end{minipage} \\
                        \hline
                        \textbf{Sequenza degli eventi principale:}
                        \begin{enumerate}
                            \item Il caso d'uso inizia quando l'utente preme il pulsante di modifica di una regia.
                            \item Il sistema consente all'utente di modificare le informazioni della regia. \label{start_mod_regia}
                            \item Il sistema richiede all'utente di confermare l'intenzione di apportare le modifiche.
                            \item \textbf{If} l'utente conferma
                                \begin{enumerate}
                                    \item Il sistema controlla la validità dei valori inseriti.
                                    \item \textbf{If} i valori sono validi 
                                        \begin{enumerate}
                                            \item Il sistema memorizza permanentemente le modifiche apportate.
                                        \end{enumerate}
                                    \item \textbf{Else}
                                        \begin{enumerate}
                                            \item Il sistema informa l'utente dell'errore.
                                            \item Il caso d'uso ricomincia da \ref{start_mod_regia}
                                        \end{enumerate}
                                \end{enumerate}
                            \item \textbf{Else}
                                \begin{enumerate}
                                    \item Il sistema ignora i cambiamenti apportati.
                                \end{enumerate}
                        \end{enumerate} \\
                        \hline
                        \cellcolor{gray!20}
                        \textbf{Postcondizioni:} \\
                        \cellcolor{gray!20}
                        \begin{minipage}{\linewidth}
                            \begin{enumerate}
                                \item Le modifiche apportate sono memorizzate e immediatamente disponibili a tutti gli utenti.
                            \end{enumerate}
                        \end{minipage} \\
                        \hline
                        \textbf{Sequenza degli eventi alternativa:} \\
                        Nessuna. \\
                        \hline
                    \end{tabular}
                \end{table}

                % UC: Elimina Regia
                \begin{table}[H]
                    \begin{tabular}{|p{\linewidth}|}
                        \hline
                        \cellcolor{gray!100}
                        \color{white}
                        \centerline{\textbf{Caso d'uso:} Elimina Regia} \\
                        \hline
                        \textbf{ID:} 31 \\
                        \hline
                        \cellcolor{gray!20}
                        \textbf{Breve descrizione:} \\
                        \cellcolor{gray!20}
                        L'utente ha premuto il pulsante di eliminazione di una regia. \\
                        \hline
                        \textbf{Attori primari:} \\
                        \begin{minipage}{\linewidth}
                            Amministratore
                        \end{minipage}
                        \vspace{-10pt} \\  % Utilizzo \vspace{} per regolare lo spazio tra le sezioni delle tabelle.
                        \hline
                        \textbf{Attori secondari:} \\
                        Nessuno. \\
                        \hline
                        \cellcolor{gray!20}
                        \textbf{Precondizioni:} \\
                        \cellcolor{gray!20}
                        \begin{minipage}{\linewidth}
                            \begin{enumerate}
                                \item L'utente ha premuto il pulsante di modifica di una regia.
                            \end{enumerate}
                        \end{minipage} \\
                        \hline
                        \textbf{Sequenza degli eventi principale:}
                        \begin{enumerate}
                            \item Il caso d'uso inizia quando l'utente preme il pulsante di eliminazione di una regia
                            \item Il sistema richiede all'utente di confermare l'intenzione di eliminare la regia.
                            \item \textbf{If} l'utente conferma
                                \begin{enumerate}
                                    \item Il sistema elimina permanentemente la regia dalla lista.
                                \end{enumerate}
                            \item \textbf{Else}
                                \begin{enumerate}
                                    \item Il sistema ignora la richiesta di eliminazione.
                                \end{enumerate}
                        \end{enumerate} \\
                        \hline
                        \cellcolor{gray!20}
                        \textbf{Postcondizioni:} \\
                        \cellcolor{gray!20}
                        \begin{minipage}{\linewidth}
                            \begin{enumerate}
                                \item Le modifiche apportate sono memorizzate e immediatamente disponibili a tutti gli utenti.
                            \end{enumerate}
                        \end{minipage} \\
                        \hline
                        \textbf{Sequenza degli eventi alternativa:} \\
                        Nessuna. \\
                        \hline
                    \end{tabular}
                \end{table}

    \section{Matrice di mapping}
        \includegraphics[height=\textheight]{imgs/matrice/matrice}

\end{document}
